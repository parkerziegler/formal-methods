\documentclass{article}
\usepackage[utf8]{inputenc}
\usepackage[margin=1in]{geometry}
\usepackage{fancyhdr}
\usepackage{listings}

\newcommand{\code}[1]{\texttt{#1}}

\title{EECS 219C: Formal Methods — Assignment 2}
\author{Parker Ziegler}
\pagestyle{fancy}
\lhead{EECS 219C: Formal Methods — Assignment 2}
\rhead{Parker Ziegler}

\begin{document}

\maketitle

\section{Sum-Sudoku}

\subsection{Bit Twiddling Hacks}

\subsubsection{(a) Equivalence of \code{f1} and \code{f2}}

\noindent My SMT-LIB encoding to check equivalence of \code{f1} and \code{f2} is located in \code{1a.ascii}. Based on my encoding, \code{f1} and \code{f2} \textbf{are not equivalent}. Z3 provides the counterexample $x = 1$, in which \code{f1(x)} evlauates to -1 while \code{f2(x)} evaluates to 1. The full output returned by Z3 is:

\begin{lstlisting}
(
  (define-fun v_1 () (_ BitVec 32)
    #x00000000)
  (define-fun x () (_ BitVec 32)
    #x00000001)
  (define-fun v_0 () (_ BitVec 32)
    #xffffffff)
  (define-fun ret_2 () (_ BitVec 32)
    #x00000001)
  (define-fun v_2 () (_ BitVec 32)
    #x00000001)
  (define-fun ret_1 () (_ BitVec 32)
    #xffffffff)
)
\end{lstlisting}

\noindent Cleaning this up to use decimal notation, the full assignment is:

\begin{lstlisting}[language=Python]
[x = 1, v_0 = -1, ret_1 = -1, v_1 = 0, v_2 = 1, ret_2 = 1]
\end{lstlisting}

\subsubsection{(b) Equivalence of \code{f3} and \code{f4}}

\noindent My SMT-LIB encoding to check equivalence of \code{f3} and \code{f4} is located in \code{1b.ascii}. Based on my encoding, \code{f3} and \code{f4} \textbf{are equivalent}.

\subsection{Formulate an SMT instance that finds a solution to Sum-Sudoku puzzles}


\end{document}