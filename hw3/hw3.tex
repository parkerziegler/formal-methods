\documentclass{article}
\usepackage[utf8]{inputenc}
\usepackage[margin=1in]{geometry}
\usepackage{fancyhdr}
\usepackage{graphicx}
\usepackage{listings}
\usepackage{parskip}

\makeatletter
  \def\@seccntformat#1{\@ifundefined{#1@cntformat}
    {\csname the#1\endcsname\space}
    {\csname #1@cntformat\endcsname}}
  \newcommand\section@cntformat{\thesection.\space}       
  \newcommand\subsection@cntformat{\thesubsection.\space}
\makeatother

\newcommand{\code}[1]{\texttt{#1}}

\renewcommand\thesubsection{\alph{subsection}}

\title{EECS 219C: Formal Methods — Assignment 2}
\author{Parker Ziegler}
\pagestyle{fancy}
\lhead{EECS 219C: Formal Methods — Assignment 2}
\rhead{Parker Ziegler}

\begin{document}

\maketitle

\section{Interrupt-Driven Program}

\subsection{Describing properties of the \code{Sys} module}

We can describe the properties of the \code{Sys} module as follows:

\begin{enumerate}
  \item \textbf{\code{invariant main\_ISR\_mutex}} — this property requires that execution of \code{main} and \code{ISR} is mutually exclusive. That is, if \code{main} is executing, \code{ISR} cannot be executing at the same time (or vice versa).
  \item \textbf{\code{property[LTL] one\_step\_ISR\_return}} — this property requires that, globally, if \code{ISR} has just returned then, in the next state, \code{ISR} will not return.
  \item \textbf{\code{property[LTL] main\_after\_ISR}} — this property requires that, globally, if \code{ISR} is currently enabled to run and, in the next state, \code{main} is enabled to run, this implies that \code{ISR} has just returned.
  \item \textbf{\code{property[LTL] ISR\_after\_main}} — this property requires that, globally, if \code{main} is enabled and, in the next state, \code{ISR} is enabled, this implies that an interrupt has occurred.
\end{enumerate}

\subsection{Interpreting counterexamples from the verifier}

Running \code{uclid} with all properties commented out \emph{except} for \code{main\_after\_ISR} results in the following counterexample:

\begin{lstlisting}
  CEX for vobj [Step #3] property main_after_ISR:safety @ IntSW.ucl, line 105
  =================================
  Step #0
    mode : main_t
    M_enable : true
    I_enable : false
    return_ISR : false
    assert_intr : initial_1570_assert_intr
  [Assertion Failure]: More than one definition found!
\end{lstlisting}

This counterexample is found for step 3 in our transition system. In this case 

Likewise, running \code{uclid} with all properties commented out \emph{except} for \code{ISR\_after\_main} results in the following counterexample:

\begin{lstlisting}
  CEX for vobj [Step #2] property ISR_after_main:safety @ IntSW.ucl, line 106
  =================================
  Step #0
    mode : main_t
    M_enable : true
    I_enable : false
    return_ISR : false
    assert_intr : false
  [Assertion Failure]: More than one definition found!
\end{lstlisting}

\end{document}